
\documentclass{article}
\usepackage{amssymb}
\usepackage{parskip}
\usepackage{graphicx} % Required for inserting images
\usepackage{pgfplots} % just 4 graphs lmao
\pgfplotsset{compat = newest}
\begin{document}
\title{Relational algebra cheat sheet}
\author{by Art. Delgadillo}
\maketitle

$Relacion=tabla$

\textbf{Los renglones duplicados se eliminan}

\textbf{Selection $\mathbf{\sigma}$:} Selecciona las tuplas que cumplen con un operador lógico.

Notación: $ \sigma _{predicado}(R)\ \ \ \ \ _{R=Relacion}$

\textbf{Projection $\mathbf{\pi}$:} Obtiene una relación a partir de R elimina los atributos que no estén en la lista de atributos.

Notación: $\pi _{lista\ de\ atributos}(R)\ \ \ \ \ _{R=Relacion}$

\vspace{5mm}
\centerline{\textbf{---Composiciones verticales---}}

\textbf{Union/Concatenación $\cup$:} Une las tuplas de manera vertical y elimina duplicados.

Notación: $R\cup S\ \ \ _{R\ y\ S\ son\ relaciones}$

Solo es realizable con relaciones compatibles.

Nota \textbf{(relaciones compatibles)}: Las relaciones deben tener el mismo grado (numero de atributos), mismo nombre de atributos y mismo dominio(valores que puede tomar cada atributo)


\textbf{Subtraction $-$:} Devuelve las tuplas que están en R pero no están en S.


Notación: $R-S\ \ \ _{R\ y\ S\ son\ relaciones}$

Solo realizable con relaciones compatibles.

\textbf{Intersection $\cap$:} Es una relación con las tuplas que están en R y también están en S al mismo tiempo.

Notación: $R\cap S\ or\ R-(R-S)\ \ \ _{R\ y\ S\ son\ relaciones}$

Solo realizable en relaciones compatibles.

\pagebreak
\centerline{\textbf{---Composiciones horizontales---}}

\textbf{Product $\times$:} Une las tuplas de manera horizontal de tal forma que cada tupla de R se concatena con cada tupla de S.

Notación: $R\times S\ \ \ _{R\ y\ S\ son\ relaciones}$

Nota: conviene renombre los atributos de R y S para evitar ambigüedad.

\textbf{Join Natural $\bowtie$:} Une las tuplas de manera horizontal de tal forma que cada tupla de R se concatena con cada tupla de S con atributos en común.

Notación: $R\bowtie S\ \ \ _{R\ y\ S\ son\ relaciones}$

Nota: Si no hay atributos en común, el resultado es el producto cartesiano.

\textbf{Theta Join $\bowtie _{condicion}$:} Une las tuplas de manera horizontal de tal forma que cada tupla de R se concatena con cada tupla de S con atributos en común y que cumplan con la condición.

Se puede construir con producto cartesiano y seleccionando las tuplas que cumplan con la condición.

Notación: $R\bowtie _{condicion} S\ \ \ _{R\ y\ S\ son\ relaciones}$

\textbf{Outer Join $\bowtie _{condicion} ^{+}$:} Une las tuplas de manera horizontal de tal forma que cada tupla de R se concatena con cada tupla de S con atributos en común y que cumplan con la condición. Además, se incluyen las tuplas de R que no tienen correspondencia en S y viceversa.

Notación: $R=\bowtie _{condicion} ^{+} S, R\bowtie _{condicion} ^{+} =S, R=\bowtie _{condicion} ^{+} =S\\$
R y S son relaciones donde se representan outer join izquierdo, derecho y completo respectivamente.

La operación de outer join por la izquierda conserva las tuplas de R que no tienen correspondencia en S.

La operación de outer join por la derecha conserva las tuplas de S que no tienen correspondencia en R.

La operación  de outer join completo conserva las tuplas de R y S que no tienen correspondencia en la otra relación.

Los atributos que no tienen correspondencia se llenan con NULL.

Útil para encontrar tuplas que no tienen correspondencia en la otra relación.

\pagebreak
\centerline{\textbf{---Operaciones sobre una relación---}}

\textbf{Rename $\rho$:} Cambia el nombre de los atributos de una relación.

Notación: $\rho _{nuevo\ nombre}(R)\ \ \ _{R=Relacion}$









\end{document}
